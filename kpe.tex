\documentclass[11pt,a4paper]{article}
\usepackage{amsmath,amsthm,amssymb,graphicx}
\usepackage{graphicx}
\usepackage{float}
\usepackage{appendix}
\usepackage{caption}
\usepackage[bookmarks=true, colorlinks, citecolor=blue, linkcolor=black]{hyperref}
\usepackage[marginal]{footmisc}
\title{Comparing Different Key-words Extration Methods}
\author{Wuhao Wang}
\date{\today}

\begin{document} % 开始写作
\maketitle


\begin{abstract}
Key-words extraction is a important task in NLP fields, many advanced tasks are based on key-words extraction.
In this project, TF-IDF, Keybert, TextRank and Yake will be compared. The dataset is from other NLP competations
and the evaluation is based on f1-score. We found that Keybert and Yake are with the highest f1-score while
TF-IDF is the most efficient method. Textrank is not flexible since it can not customize the number of key-words.  
\end{abstract}
\section{Introduction}

Key-words detecing is always a important task in many text applications. In this project, four methods will be discussed and compared.
The first methods is TF-IDF$\footnote{\url{https://en.wikipedia.org/wiki/Tf/$\%$E2/$\%$80$\%$93idf}}$, which basically calculate and compare
the frequency of a specific word in all the text data and one text data. Higher TF-IDF value means higher importance of this word to the text. Another model is
known as Yake\cite{yake1}\cite{yake2}\cite{yake3}. Yake can be condidered as an extension of TF-IDF, this algorithm will calculate TF-IDF value and also cosine similarity, then a word graph will be 
constructed to obtain the weights of each candidate word. The third one is a self-supervised 
learning method called KeyBert \cite{keybert}. In this approach, document-level vector representations 
are extracted by BERT. Subsequently, word vectors are extracted for the N-gram words/phrases, and then, cosine similarity will be applied to find the most similar 
word/phrase to the document. Finally, the selected words will be identified as the words that best describe the entire document. Last Algorithm is TextRank\cite{textrank}.
The TextRank algorithm is a graph-based sorting algorithm. The work flows are as follows:putting text segmentation into several constituent units (e.g. sentences), 
building a node connection graph, using similarity between sentences as the edge Weights, calculate the TextRank value of the sentence through loop iterations, and finally extract the sentences with high rankings 
and combine them into text summaries.
   % \subsection{Motivation}
   % When thinking about the text mining project, the most common topics are spam detection, text emotion prediction, search results matching, etc. And all these three task
   % require some keywords to get prediction. So, we can abstract such tasks into two parts. The first one is keywords extraction and another one is the application of keywords.
   % The keywords extraction can be a start point of piplines in the text mining project, and optimize this step would be very necessary if someone wants to improve the prediction
   % results. Under such circumstance, I would like to try and compare currently popular keywords extraction methods and find out which of them are better in terms of quality and time efficiency.
\newpage
\section{Related work}

Mohammad\text{-}Reza Feizi\text{-}Derakhshi\cite{bmo} used bert model to do key-phrase extraction. The bert model with graph-text-based embedding can reach 0.70 f1-score. 
MingXi Zhang\cite{article33} tested Textrank algorithm in many key-words extraction dataset, but they only have the precision as evaluation.
In 2018, Bijoyan Das and Sarit Chakraborty\cite{DBLP} used TF-IDF to grab key-words and use those key-words to do different tasks. In a classification task, they reach 96.83$\%$
accuracy. Shahzad Qaiser and Ramsha Ali \cite{234}used TF-IDF to find the words with highest
relevance to the documents , they compared their rank results with the golden label but they did not mention evaluation score like f1-score


\section{Data Set}
The data set used in this project is Inspec \cite{inspec}. The language is English, the original text is collected from scientific paper within computer
science fields.Below is the numeric features of the data.

\begin{center}
\begin{tabular}{cccc}
    \hline
    N-docs& total Goldkey& Tokens per doc& Absent Goldkey\\
    \hline
    2000&29230& 128.2& 37.7\\
    
    \hline
    \end{tabular}
\end{center}
\noindent
\textbf{N-docs} is the number of documents contained in the data, \textbf{total Goldkey} is the number
of gold key-words in the dataset, \textbf{tokens per doc} is average number of tokens in each document,
\textbf{Absent Godkey} is the percentage of gold key-words that absent in the documents.


\section{Evaluation}
In the competation \cite{augenstein-etal-2017-semeval}hold by Association for Computational Linguistics in Vancouver, Canada,
they use f1-score to evaluate the results. In this project, all the model performance will be evaluated by f1-score in this project.
The f1-score can be calculated by:
$$
\begin{aligned}
f1 &= \frac{2}{recall^-1+precision^-1} \\
&= 2\ \frac{precision \times recall}{precision+recall} \\
&= \frac{2tp}{2tp+fp+fn}
\end{aligned}
$$
In this project, \textbf{true positive} means if a gold key-word is predicted as a gold key-word,
\textbf{false negative} means if a gold key-word is predicted as a non gold key-word,\textbf{false positive}
means if a non gold key-word is predicted as a gold key-word.
\section{Experiment}
In this section, the results of each method will be presented and discussed.
\subsection{TF-IDF}
For TF-IDF method, this project is referring to spacy package$\footnote{\url{https://github.com/explosion/spaCy}}$.
The results of TF-IDF method are as follows. N is number of key-words that model would extract, f1 is the corresponding
f1-score.
\begin{center}
    \begin{tabular}{cccccc}
        \hline
        N& 10& 12& 16& 18& 20\\
        \hline
        f1& 0.226& 0.242& 0.281& 0.266& 0.239 \\
        \hline
    \end{tabular}
\end{center}

\subsection{Yake}
For yake model, this project is referring to official package$\footnote{\url{https://github.com/LIAAD/yake}}$.
The results of yake are as follows. N is number of key-words that model would extract, f1 is the corresponding
f1-score. Note that there are many other parameters can be tuned, see appendix for more details.
\begin{center}
    \begin{tabular}{cccccc}
        \hline
        N& 10& 12& 16& 18& 20\\
        \hline
        f1& 0.291& 0.327& 0.381& 0.367& 0.345 \\
        \hline
    \end{tabular}
\end{center}

\subsection{Keybert}
For Keybert model, this project is referring to official package$\footnote{\url{https://github.com/MaartenGr/KeyBERT}}$.
The results of Keybert are as follows. N is number of key-words that model would extract, f1 is the corresponding
f1-score. Note that there are many other parameters can be tuned, see appendix for more details.
\begin{center}
    \begin{tabular}{cccccc}
        \hline
        N& 10& 12& 16& 18& 20\\
        \hline
        f1& 0.268& 0.284& 0.350& 0.301& 0.238 \\
        \hline
    \end{tabular}
\end{center}

\subsection{Textrank}
For textrank model, this project is referring to two different implementations,one is summaNLP$\footnote{\url{https://github.com/summanlp/textrank}}$,
another one is spacy $\footnote{\url{https://spacy.io/universe/project/spacy-pytextrank}}$. Textrank algorithm does not
offer "number of key-words" parameter, the results of two different implementations are as follows:

\begin{center}
    \begin{tabular}{ccc}
        \hline
        implementations& summaNLP& Spacy\\
        \hline
        f1& 0.200& 0.179 \\
        \hline
    \end{tabular}
\end{center}
\noindent

\section{Discussion}
Yake is similar to TF-IDF, the weights of each word comes from document-level statistics. But Yake considers
more features than TF-IDF and the f1-score also reflects that Yake is more advanced than TF-IDF.
\vspace{11pt}
\noindent
Keybert uses BERT to extract the word vectors and considers similarity between words and documents. So, the result is
highly depends on which pre-trained network is used. 
\vspace{11pt}
\noindent
Textrank is based on graph algorithm. Like transformer-based model, it will divide text into different part. The different
thing is that Textrank will consider those elements as points and edges in a graph. Then a voting system will
select the most important words.

\section{Conclusion}
From the experiment results, one can easily see that yake and keybert are much better than other two methods. For this 
dataset, every document contains averagely 15 golden key-words. And for textrank which does not receive "number of key-words" parameter,
the f1-score is lower because of too few key-words prediction.In this project, We also try to set "number of key-words = 5" for other
models and results are quite similar to textrank.In one word,TF-IDF methods is the fastest method among these four methods while keybert and yake have the highest f1-score.
\vspace{11pt}
\noindent
Note that in Inspec dataset, 37.7$\%$ of key-words do not show up in the text, which means the 'fn' in the f1-score is atleast
0.37. That is a important reason why all the models can not have high f1-score.
\vspace{11pt}
\noindent
The drawback of TF-IDF is obvious. TF-IDF methodcan only get every single key-word. When it comes
to key phrase extraction(KPE), TF-IDF method can not work. In this project, the dataset semeval2017 \cite{augenstein-etal-2017-semeval} 
,which contains key phrase,is also applied to all of these four models. We pick the same hyperparameters as in the Inspec datasetm, but the length
of key phrase is set in a range from 1 to 4.
\vspace{11pt}
\begin{center}
    \begin{tabular}{ccccc}
        \hline
        models& TF-IDF& Keybert& TextRank& Yake\\
        \hline
        f1& 0.04& 0.13& 0.12& 0.15 \\
        \hline
    \end{tabular}
\end{center}
\noindent

\section{Future Work}
Based on this project, there are multiple future tasks. The most important thing is thinking how can
TF-IDF method predict key phrase? Maybe this can be achieved by combining the candidate key-words and re-calculate
the weights. Another thinking is how to improve other models' f1-score.
\vspace{11pt}
\noindent
In the semeval-2017 competation, the highest team reach the f1-score of 0.68. It is a quite good results but is still
far away from the practical usage standard.

\section{Appendix}
\appendix
\section{Hyperparameter}
For Keybert
\begin{center}
    \begin{tabular}{cccc}
        \hline
        hp& diversity&use$\_$maxsum & model\\
        \hline
        value& 0.2& False& 'all-MiniLM-L6-v2' \\
        \hline
    \end{tabular}
\end{center}
\noindent
For Yake
\begin{center}
    \begin{tabular}{cccc}
        \hline
        hp& deduplication$\_$threshold& deduplication$\_$algo& windowSize\\
        \hline
        value&  0.9& 'seqm'& 1 \\
        \hline
    \end{tabular}
\end{center}
\bibliographystyle{plain}
\bibliography{book}

\end{document}